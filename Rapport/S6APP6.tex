% ======================================================================


% Sujet du document
% Informations importantes
%
%
% Prénom Nom
% H. Dube 2020
% ======================================================================
% Ce code rassemble les efforts d'étudiants de la faculté de génie  
% de l'université de Sherbrooke afin de faire un template LaTeX moderne
% dédié à l'écriture de rapport universitaire.
% Ce document est libre d'être utilisé et modifié.
% ======================================================================

% ----------------------------------------------------
% Initialisation
% ----------------------------------------------------
\documentclass{udes_rapport} % Voir udes_rapport.cls
\usepackage{amsmath}

\numberwithin{equation}{section}
\begin{document}
\selectlanguage{french}

% ----------------------------------------------------
% Configurer la page titre
% ----------------------------------------------------

% Information
\faculte{Génie}
\departement{génie électrique et génie informatique}
\app{6}{Propagation guidée d'ondes electromagnétiques}
\professeur{François Boone}
\etudiants{Hubert Dubé - dubh3401 \\ Gabriel Lavoie - lavg2007}
\dateRemise{31 juillet 2020}


% ======================================================================
\pagenumbering{roman} % met les numéros de pages en romain
% ----------------------------------------------------
% Page titre
% ----------------------------------------------------
%\fairePageTitre{LOGO} % Options: [STD, LOGO]
\begin{titlepage}
\fairePageTitre{LOGO}
\end{titlepage}
\newpage

% ----------------------------------------------------
% Table des matières
% ----------------------------------------------------
\tableofcontents
\newpage
% ----------------------------------------------------
% Table des figures
% ----------------------------------------------------
\listoffigures
\newpage

% ======================================================================
% Document
% ======================================================================
\pagenumbering{arabic} % met des chiffres arabes
\setcounter{page}{1} % reset les numéros de pages

\section{Une seule ligne}
	% Circuit à une seule ligne
\subsection{Démarche analytique}
La première étape d'analyse consiste à trouver les coefficients de rebonds afin de pouvoir faire le diagramme de rebond (\textit{bounce diagramm}).
Avec la formule 6.25 de Paul :
\begin{equation}
\Gamma_s = \frac{R_s - Z_c}{R_s + Z_c} = \frac{-2}{3}
\end{equation}
De la même façon, le coefficient de la charge est déterminé.
Comme la charge n'est qu'un circuit ouvert, $R_L = \infty$.
\begin{equation}
\Gamma_L = \frac{R_L - Z_c}{R_L + Z_c} = 1
\end{equation}
D'après le résultat de $\Gamma_L$ il est attendu que le retour de l'onde soit toujours de la même valeur que l'onde envoyée. Pour ce qui est de $\Gamma_s$, comme il est négatif, la tension aura donc tendance à osciller vue que la valeur que l'onde renvoyée changera entre positive et négative à chaque rebond.
Le diagramme de rebond se retrouve dans l'annexe \ref{Ann:bounce}. 
Les résultats pour chaque instant $T$ sont dans le tableaux suivant :
\begin{center}
\captionof{table}{Valeurs écahntillonnées des rebonds}
\label{tab:bounce}
\begin{tabular}{c*{2}c| c*{4}c}
\toprule
Temps (ns)	&V1(V)	&V2(V)	&Temps (ns)	&V1(V)	&V2(V)\\ 
\midrule
\midrule
0	&4.2	&0		&7	&-0.3	&-4.3\\
1	&4.2	&8.3	&8	&-0.7	&1.2\\
2	&5.6	&8.3	&9	&0.2	&2.9\\
3	&5.6	&2.8	&10	&4.7	&-1.7\\
4	&4.6	&2.8 	&11	&4.0	&6.7\\
5	&0.5	&6.5	&12	&5.5	&9.4\\
6	&1.1	&-1.9	&-	&-		&-\\
\bottomrule
\end{tabular}
\end{center}

%%%%%%%% FIGURE %%%%%%%
%simple figure
\insertFigure{simple}{one_line/one_line_1}{0.75}{Résultat de simulation pour la ligne idéale de base}
%%%%%%%%%%%%%%%%%%%%%%


%%%%%%%% FIGURE %%%%%%%
%simple figure
\insertFigure{simple}{one_line/one_line_delay_0p5}{0.75}{Résultat de simulation pour la ligne idéale avec $T = 0.5$ns}
%%%%%%%%%%%%%%%%%%%%%%

%%%%%%%% FIGURE %%%%%%%
%simple figure
\insertFigure{simple}{one_line/one_line_delay_0p1}{0.75}{Résultat de simulation pour la ligne idéale avec $T= 0.1$ns}
%%%%%%%%%%%%%%%%%%%%%%

%%%%%%%% FIGURE %%%%%%%
%simple figure
\insertFigure{simple}{one_line/one_line_rise_fall}{0.75}{Résultat de simulation pour la ligne idéale avec $T= 0.1$ns, $\tau_r = \tau_f = 0.5$ns}
%%%%%%%%%%%%%%%%%%%%%%
\section{Deux lignes couplées}
	% circuit à deux lignes
Avec un circuit à deux lignes et un plan de masse, deux modes de propagation sont présents. On détermine donc, à partir de l'outil \texttt{linecalc}, les impédances $Z_e$ et $Z_o$ qui permettent de conserver une impédance caractéristique de $50\,\omega$.
\[Z_e = 53.28\quad Z_o = 49.85\]
\[Z_c = \sqrt{Z_eZ_c} = 51.54\]

La longueur électrique de la ligne est de $22.19°$. En terme de longeur d'onde, ceci représente une longueur électrique de $0.0616\lambda$. Avec une période de $10\,ns$, la source a une fréquence fondamentale de $100\,MHz$. 
\\
En comparaison avec la ligne simple, on remarque que l'effet capacitif du couplage avec la deuxième ligne amplifie les harmoniques du signal à $V_2$. On voit cet effet dans les pics lors de changement rapides d'amplitude.
%%%%%%%% FIGURE %%%%%%%
%simple figure
\insertFigure{simple}{two_lines/base_G}{0.75}{Résultat de simulation pour la ligne 50 $\Omega$}
%%%%%%%%%%%%%%%%%%%%%%
%\begin{comment}


\subsection{Simulation avec variation de signal}
Dans l'analyse des signaux à $V_3$ et $V_4$, on remarque premièrement que les deux signaux ont presque exactement la même forme, et suivent les directions de $V_2$. Les signaux sont centrés sur 0, ce qui s'explique par le fait qu'ils sont induits par un couplage capacitif, qui supprime la composante DC. 
\\
À partir de cette simulation, on fait varier le temps de montée et de descente du signal. On remarque encore que les harmoniques à haute fréquences sont réduites. 
\\
Un simulation avec une longueur électrique de 360 est ensuite réalisée. Comme on peut s'y attendre, le rebond à la sortie de la ligne arrive en même temps que la prochaine impulsion. En diaphonie, le résultat est deux trains d'impulsions identiques séparés par une période.

%%%%%%% FIGURE %%%%%%%
%double figure
\makebox[\textwidth][c]{
\noindent\begin{minipage}{1.3\textwidth}
	\begin{minipage}{0.5\textwidth}
		\insertFigure{double}{two_lines/base_G}{0.9}{void}
	\end{minipage}%
	\begin{minipage}{0.5\textwidth}
		\insertFigure{double}{two_lines/base_R}{1}{void}
	\end{minipage} 
\captionof{figure}{Observations initials} 
\label{fig:G_R_init} 
\end{minipage}
}
%%%%%%%%%%%%%%%%%%%%%%

%%%%%%% FIGURE %%%%%%%
%double figure
\makebox[\textwidth][c]{
\noindent\begin{minipage}{1.3\textwidth}
	\begin{minipage}{0.5\textwidth}
		\insertFigure{double}{two_lines/G_rise_fall}{1}{void}
	\end{minipage}%
	\begin{minipage}{0.5\textwidth}
		\insertFigure{double}{two_lines/R_rise_fall}{0.9}{void}
	\end{minipage} 
\captionof{figure}{Changement pour $\tau_r = \tau_f = 0.5$ns} 
\label{fig:G_R_rise_fall} 
\end{minipage}
}
%%%%%%%%%%%%%%%%%%%%%%

%%%%%%% FIGURE %%%%%%%
%double figure
\makebox[\textwidth][c]{
\noindent\begin{minipage}{1.3\textwidth}
	\begin{minipage}{0.5\textwidth}
		\insertFigure{double}{two_lines/G_E}{1}{void}
	\end{minipage}%
	\begin{minipage}{0.5\textwidth}
		\insertFigure{double}{two_lines/R_E}{1}{void}
	\end{minipage} 
\captionof{figure}{Changement de la longueur electrique pour E = 360} 
\label{fig:G_R_E} 
\end{minipage}
}
%%%%%%%%%%%%%%%%%%%%%%

\section{Adaptation d'impédances}
	% matching d'impedance, circuit a 2 lignes

\section{Correction de circuit non idéal}
	% corrections du circuit


\end{document}













