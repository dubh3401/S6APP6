% circuit à deux lignes
\subsection{Caractéristiques de ligne pour avoir }%$Z_C = 50\Omega$}
\note{comparer avec les résultats de la lignes simple}

%%%%%%%% FIGURE %%%%%%%
%simple figure
\insertFigure{simple}{two_lines/base_G}{0.75}{Résultat de simulation pour la ligne 50 $\Omega$}
%%%%%%%%%%%%%%%%%%%%%%
%\begin{comment}
\subsection{Longueur électrique et fréquence foncamentale}

\subsection{Simulation avec variation de signal}
\note{Analyse de simulation V3 et V4, et ensuite comparaison entre tous les graphiques}

%%%%%%% FIGURE %%%%%%%
%double figure
\makebox[\textwidth][c]{
\noindent\begin{minipage}{1.3\textwidth}
	\begin{minipage}{0.5\textwidth}
		\insertFigure{double}{two_lines/base_G}{0.9}{void}
	\end{minipage}%
	\begin{minipage}{0.5\textwidth}
		\insertFigure{double}{two_lines/base_R}{1}{void}
	\end{minipage} 
\captionof{figure}{Observations initials} 
\label{fig:G_R_init} 
\end{minipage}
}
%%%%%%%%%%%%%%%%%%%%%%

%%%%%%% FIGURE %%%%%%%
%double figure
\makebox[\textwidth][c]{
\noindent\begin{minipage}{1.3\textwidth}
	\begin{minipage}{0.5\textwidth}
		\insertFigure{double}{two_lines/G_rise_fall}{1}{void}
	\end{minipage}%
	\begin{minipage}{0.5\textwidth}
		\insertFigure{double}{two_lines/R_rise_fall}{0.9}{void}
	\end{minipage} 
\captionof{figure}{Changement pour $\tau_r = \tau_f = 0.5$ns} 
\label{fig:G_R_rise_fall} 
\end{minipage}
}
%%%%%%%%%%%%%%%%%%%%%%

%%%%%%% FIGURE %%%%%%%
%double figure
\makebox[\textwidth][c]{
\noindent\begin{minipage}{1.3\textwidth}
	\begin{minipage}{0.5\textwidth}
		\insertFigure{double}{two_lines/G_E}{1}{void}
	\end{minipage}%
	\begin{minipage}{0.5\textwidth}
		\insertFigure{double}{two_lines/R_E}{1}{void}
	\end{minipage} 
\captionof{figure}{Changement de la longueur electrique pour E = 360} 
\label{fig:G_R_E} 
\end{minipage}
}
%%%%%%%%%%%%%%%%%%%%%%
