% circuit à deux lignes
Avec un circuit à deux lignes et un plan de masse, deux modes de propagation sont présents. On détermine donc, à partir de l'outil \texttt{linecalc}, les impédances $Z_e$ et $Z_o$ qui permettent de conserver une impédance caractéristique de $50\,\omega$.
\[Z_e = 53.28\quad Z_o = 49.85\]
\[Z_c = \sqrt{Z_eZ_c} = 51.54\]

La longueur électrique de la ligne est de $22.19°$. En terme de longeur d'onde, ceci représente une longueur électrique de $0.0616\lambda$. Avec une période de $10\,ns$, la source a une fréquence fondamentale de $100\,MHz$. 
\\
En comparaison avec la ligne simple, on remarque que l'effet capacitif du couplage avec la deuxième ligne amplifie les harmoniques du signal à $V_2$. On voit cet effet dans les pics lors de changement rapides d'amplitude.
%%%%%%%% FIGURE %%%%%%%
%simple figure
\insertFigure{simple}{two_lines/base_G}{0.75}{Résultat de simulation pour la ligne 50 $\Omega$}
%%%%%%%%%%%%%%%%%%%%%%
%\begin{comment}


\subsection{Simulation avec variation de signal}
Dans l'analyse des signaux à $V_3$ et $V_4$, on remarque premièrement que les deux signaux ont presque exactement la même forme, et suivent les directions de $V_2$. Les signaux sont centrés sur 0, ce qui s'explique par le fait qu'ils sont induits par un couplage capacitif, qui supprime la composante DC. 
\\
À partir de cette simulation, on fait varier le temps de montée et de descente du signal. On remarque encore que les harmoniques à haute fréquences sont réduites. 
\\
Un simulation avec une longueur électrique de 360 est ensuite réalisée. Comme on peut s'y attendre, le rebond à la sortie de la ligne arrive en même temps que la prochaine impulsion. En diaphonie, le résultat est deux trains d'impulsions identiques séparés par une période.

%%%%%%% FIGURE %%%%%%%
%double figure
\makebox[\textwidth][c]{
\noindent\begin{minipage}{1.3\textwidth}
	\begin{minipage}{0.5\textwidth}
		\insertFigure{double}{two_lines/base_G}{0.9}{void}
	\end{minipage}%
	\begin{minipage}{0.5\textwidth}
		\insertFigure{double}{two_lines/base_R}{1}{void}
	\end{minipage} 
\captionof{figure}{Observations initials} 
\label{fig:G_R_init} 
\end{minipage}
}
%%%%%%%%%%%%%%%%%%%%%%

%%%%%%% FIGURE %%%%%%%
%double figure
\makebox[\textwidth][c]{
\noindent\begin{minipage}{1.3\textwidth}
	\begin{minipage}{0.5\textwidth}
		\insertFigure{double}{two_lines/G_rise_fall}{1}{void}
	\end{minipage}%
	\begin{minipage}{0.5\textwidth}
		\insertFigure{double}{two_lines/R_rise_fall}{0.9}{void}
	\end{minipage} 
\captionof{figure}{Changement pour $\tau_r = \tau_f = 0.5$ns} 
\label{fig:G_R_rise_fall} 
\end{minipage}
}
%%%%%%%%%%%%%%%%%%%%%%

%%%%%%% FIGURE %%%%%%%
%double figure
\makebox[\textwidth][c]{
\noindent\begin{minipage}{1.3\textwidth}
	\begin{minipage}{0.5\textwidth}
		\insertFigure{double}{two_lines/G_E}{1}{void}
	\end{minipage}%
	\begin{minipage}{0.5\textwidth}
		\insertFigure{double}{two_lines/R_E}{1}{void}
	\end{minipage} 
\captionof{figure}{Changement de la longueur electrique pour E = 360} 
\label{fig:G_R_E} 
\end{minipage}
}
%%%%%%%%%%%%%%%%%%%%%%
