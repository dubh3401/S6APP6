% Circuit à une seule ligne
\subsection{Démarche analytique}
La première étape d'analyse consiste à trouver les coefficients de rebonds afin de pouvoir faire le diagramme de rebond (\textit{bounce diagramm}).
Avec la formule 6.25 de Paul :
\begin{equation}
\Gamma_s = \frac{R_s - Z_c}{R_s + Z_c} = \frac{-2}{3}
\end{equation}
De la même façon, le coefficient de la charge est déterminé.
Comme la charge n'est qu'un circuit ouvert, $R_L = \infty$.
\begin{equation}
\Gamma_L = \frac{R_L - Z_c}{R_L + Z_c} = 1
\end{equation}
D'après le résultat de $\Gamma_L$ il est attendu que le retour de l'onde soit toujours de la même valeur que l'onde envoyée. Pour ce qui est de $\Gamma_s$, comme il est négatif, la tension aura donc tendance à osciller vue que la valeur que l'onde renvoyée changera entre positive et négative à chaque rebond.
Le diagramme de rebond se retrouve dans l'annexe \ref{Ann:bounce}. 
Les résultats pour chaque instant $T$ sont dans le tableaux suivant :
\begin{center}
\captionof{table}{Valeurs écahntillonnées des rebonds}
\label{tab:bounce}
\begin{tabular}{c*{2}c| c*{4}c}
\toprule
Temps (ns)	&V1(V)	&V2(V)	&Temps (ns)	&V1(V)	&V2(V)\\ 
\midrule
\midrule
0	&4.2	&0		&7	&-0.3	&-4.3\\
1	&4.2	&8.3	&8	&-0.7	&1.2\\
2	&5.6	&8.3	&9	&0.2	&2.9\\
3	&5.6	&2.8	&10	&4.7	&-1.7\\
4	&4.6	&2.8 	&11	&4.0	&6.7\\
5	&0.5	&6.5	&12	&5.5	&9.4\\
6	&1.1	&-1.9	&-	&-		&-\\
\bottomrule
\end{tabular}
\end{center}

%%%%%%%% FIGURE %%%%%%%
%simple figure
\insertFigure{simple}{one_line/one_line_1}{0.75}{Résultat de simulation pour la ligne idéale de base}
%%%%%%%%%%%%%%%%%%%%%%


%%%%%%%% FIGURE %%%%%%%
%simple figure
\insertFigure{simple}{one_line/one_line_delay_0p5}{0.75}{Résultat de simulation pour la ligne idéale avec $T = 0.5$ns}
%%%%%%%%%%%%%%%%%%%%%%

%%%%%%%% FIGURE %%%%%%%
%simple figure
\insertFigure{simple}{one_line/one_line_delay_0p1}{0.75}{Résultat de simulation pour la ligne idéale avec $T= 0.1$ns}
%%%%%%%%%%%%%%%%%%%%%%

%%%%%%%% FIGURE %%%%%%%
%simple figure
\insertFigure{simple}{one_line/one_line_rise_fall}{0.75}{Résultat de simulation pour la ligne idéale avec $T= 0.1$ns, $\tau_r = \tau_f = 0.5$ns}
%%%%%%%%%%%%%%%%%%%%%%